\section{Using pbdPAPI}
\label{sec:use}


The \thispackage package has 2 distinct interfaces for gathering performance 
data, one low-level and ``one'' high-level.  The low-level interface is 
reminiscent of \PAPI's own native \C interface, and ``the'' high-level interface 
is a collection of similar-looking wrappers around this interface.

Note that not all utilities will function on all hardware platforms.  Those 
which are not supported for a given hardware platform will return an error.  
You can check which profiler events are available on your hardware platform 
using the \thispackage utility \code{papi.avail()} called with no argument.


\subsection{High-Level Interface}

We offer numerous useful simplified interfaces for profiling \R code, shown in 
the table below:
\begin{center}
\vspace{0.2cm}
\begin{tabular}{ll} \hline\hline
Function & Description of Measurement \\ \hline
\code{system.cache()} & Cache misses, hits, accesses, and reads \\
\code{system.cpuormem()} & Characterize code as CPU or RAM bound (see docs 
for definition) \\
\code{system.epc()} & Events per cycle \\
\code{system.flips()} & Time, floating point instructions, and Mflips \\
\code{system.flops()} & Time, floating point operations, and Mflops \\
\code{system.idle()} & Idle cycles \\
\code{system.utilization()} & CPU utilization (see documentation for 
definition) \\
\code{ipcm.cache()} & Cache misses and hits for Intel PCM users \\
\hline\hline
\end{tabular}
\vspace{0.2cm}
\end{center}

Each of these utilities behaves somewhat like \R's own \code{system.time()} 
function.  Simply pass a valid \R expression in for the \code{expr} argument 
for any of these functions and the appropriate measurement will be returned.

See \secref{sec:examples} and/or the package demos for some example usage.


\subsection{Low-Level Interface}

The low-level interface is provided for users who more familiar with hardware and may have 
questions about the operation of a block of code that do not easily fall into the the set of 
wrappers that make up the high-level interface.  Even this comes in two forms.  The first is the 
\code{system.event()} interface, which looks like the high-level interface functions, except that 
you must explicitly supply a \PAPI events vector, with event names stored as strings.  For a list 
of all possible events, and how to determine which are supported on your platform, see the \R help 
\code{?papi.avail}.

As an example, suppose you wanted to measure the level 1 data cache misses and L1 data cache hits. 
After looking up the names for these events in the help file mentioned just above, you would simply 
call
\begin{lstlisting}[language=rr]
events <- c("PAPI_L1_DCM", "PAPI_L1_DCH")
system.event(1+1, events=events)
\end{lstlisting}

Here, replace \code{1+1} with your preferred computation.

The final form of the low-level interface is essentially a deconstruction of the code making up the 
body of the \code{system.event()} function.  This would allow you to skip several forms of error 
checking (if you want to do that for some reason).  Primarily its purpose is to resemble the 
native \C-level \PAPI interface.  We really don't recommend you use this, but it's there if you 
really believe you need it.

Revisiting the above example, you would call:
\begin{lstlisting}[language=rr]
events <- c("PAPI_L1_DCM", "PAPI_L1_DCH")

papi.start(events=events)
1+1
papi.stop(events=events)
\end{lstlisting}


\
