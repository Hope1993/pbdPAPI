\section{Installation}
\label{sec:installation}

In this section, we will describe the various ways that one can install the \thispackage package.



\subsection{WITHOUT a System Installation of PAPI}

This is the default method of installation.  Here, the \PAPI library will automatically be built first as a static library, and then the \thispackage package will be built and linked against that static library.  All of this is handled completely transparently, and should only go wrong if your system is not supported by \PAPI.  This is the simplest approach, and should cover most users.  Simply build the package as you would any other:
\begin{Command}
R CMD INSTALL pbdPAPI_0.1-0.tar.gz
\end{Command}
and using the \pkg{devtools} package:
\begin{lstlisting}
library(devtools)
install_github(username="wrathematics", repo="pbdPAPI")
\end{lstlisting}




\subsection{WITH an Existing System Installation of PAPI}

If you already have a system installation of \PAPI available, it makes more sense to link with that existing library.  The one catch is that the static library \emph{must} have been compiled with \code{-fPIC}, which is non-standard.  To build an external \PAPI library in this way, you should do so by first setting:
\begin{Command}
export CC="${CC} -fPIC"
\end{Command}
Assuming that \code{CC} is set; if not, you can use \code{cc} in the right hand side.

To link with an external installation of \PAPI, from the command line, execute:
\begin{Command}
R CMD INSTALL pbdPAPI_0.1-0.tar.gz \ 
    --configure-args="--enable-system-papi \ 
    --with-papi-home=location/of/PAPI/install"
\end{Command}
and using the \pkg{devtools} package:
\begin{lstlisting}
library(devtools)
install_github(username="wrathematics", repo="pbdPAPI", 
    args="--configure-args='--enable-system-papi 
          --with-papi-home=location/of/PAPI/install'")
\end{lstlisting}
