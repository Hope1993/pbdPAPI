\section{Installation}
\label{sec:installation}

In this section, we will describe the various ways that one can install the 
\thispackage package.

Please be aware that the full features of \thispackage are not supported on
Windows or Mac at this time.  This is because the \PAPI library itself does
not support these platforms.  Until \PAPI officially supports these platforms,
an alternative option is provided for some Intel CPUs in \thispackage through
Intel Performance Counter Monitor \href{http://www.intel.com/software/pcm}.
If we become able to support more platforms, we will do so 
immediately.

For a complete list of supported platforms for PAPI, see 
\href{http://icl.cs.utk.edu/papi/custom/index.html?lid=62&slid=96}%
{the official PAPI documentation}.



\subsection{Important Notes}

It is possible for \PAPI and/or \thispackage to build, but still have no access
to any counter information.  For example, the package will build on the popular 
ARM architecture device 
Raspberry Pi\footnote{\url{http://www.raspberrypi.org/}}.
However, the device does not have any hardware counters, and as such, 
\thispackage effectively can not be used.  If the package installs without 
error, you can see how many hardware counters you have available by calling:
\begin{lstlisting}[language=rr]
library(pbdPAPI)
system.ncounters()
\end{lstlisting}

However, this too has some caveats.  Those with Intel Sandy Bridge and/or Intel 
Ivy Bridge architectures should be aware that flops counts are unreliable on 
these platforms.  This is a problem with the hardware returning incorrect 
values, not with \PAPI or \thispackage.  For more details, see: 
\href{http://icl.cs.utk.edu/projects/papi/wiki/PAPITopics:SandyFlops}%
{the \PAPI documentation} on this matter.


Additionally, not all hardware platforms support all counters.  For the 
high-level 
interface, this simply means that some things you may wish to try simply can 
not physically work, and as such will return an error.  

For a detailed list of all counters and whether or not your particular platform 
supports them, simply call 
\begin{lstlisting}[language=rr]
library(pbdPAPI)
papi.avail()
\end{lstlisting}





\subsection{Installing WITHOUT a System Installation of PAPI}

This is the default method of installation.  Here, the \PAPI library will  
automatically be built first as a static library, and then the \thispackage 
package will be built and linked against that static library.  All of this is 
handled completely transparently, and should only go wrong if your system is not 
supported by \PAPI.  This is the simplest approach, and should cover most users. 
 Simply build the package as you would any other:
\begin{Command}
R CMD INSTALL pbdPAPI_0.1-0.tar.gz
\end{Command}
and using the \pkg{devtools} package:
\begin{lstlisting}[language=rr]
library(devtools)
install_github(username="wrathematics", repo="pbdPAPI")
\end{lstlisting}




\subsection{Installing WITH an Existing System Installation of PAPI}

If you already have a system installation of \PAPI available, it makes more  
sense to link with that existing library.  The one catch is that the static 
library \emph{must} have been compiled with \code{-fPIC}, which is 
non-standard. Not all versions of \PAPI are supported.  \thispackage package 
was built and tested with \PAPI \PAPIversion.


 To build an external \PAPI library in this way, you should do so by first 
setting:
\begin{Command}
export CC="${CC} -fPIC"
\end{Command}
Assuming that \code{CC} is set; if not, you can use \code{cc} in the right hand 
side.

To link with an external installation of \PAPI, from the command line, execute:
\begin{Command}
R CMD INSTALL pbdPAPI_0.1-0.tar.gz \ 
    --configure-args="--enable-system-papi \ 
    --with-papi-home=location/of/PAPI/install"
\end{Command}
and using the \pkg{devtools} package:
\begin{lstlisting}
library(devtools)
install_github(username="wrathematics", repo="pbdPAPI", 
    args="--configure-args='--enable-system-papi 
          --with-papi-home=location/of/PAPI/install'")
\end{lstlisting}

\subsection{Installing with Intel PCM}

If you are using an Intel CPU on a platform unsupported by \PAPI, you may try using 
\IPCM with \thispackage.  The \IPCM software requires additional drivers to function.
On Linux and FreeBSD, these drivers are typically already available in standard kernels.
Required drivers for OS X and Windows are built automatically when this package is built.
Please also note that root or administrator privileges may be required for building,
installing, and running \thispackage with \IPCM enabled.

\begin{Command}
R CMD INSTALL pbdPAPI_0.1-0.tar.gz \ 
    --configure-args=--enable-ipcm \ 
    --no-test-load
\end{Command}
and using the \pkg{devtools} package:
\begin{lstlisting}
library(devtools)
install_github(username="wrathematics", repo="pbdPAPI", 
    args="--configure-args=--enable-ipcm")
\end{lstlisting}

Running \thispackage may require setting the LD\_LIBRARY\_PATH environment variable to
include the path where the \IPCM library has been installed.
\begin{Command}
	export LD_LIBRARY_PATH=\$LD_LIBRARY_PATH:/usr/local/lib/R/site-library/pbdPAPI/libs
\end{Command}
