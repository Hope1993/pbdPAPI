\section{Introduction}
\label{sec:introduction}

The value of profiling code is indisputable; if you disagree, stop pretending  
that you actually care about data.  \R's own \code{Rprof()} function is 
extremely useful, but its profiling capabilities are limited to simple timings 
of \R functions. This is a very good starting point in performance analysis and 
can quickly help the \R programmer focus in on bottlenecks.  But for more 
experienced developers (especially those working with compiled code) additional 
performance information can be invaluable. Access to low-level hardware counter 
data can have tremendous impact when trying to understand and optimize 
performance of compiled code. 

The \thispackage~\citep{Schmidt2014pbdPAPIpackage} package offers access to  
this low-level hardware counter information by way of the high-level \C library 
\PAPI~\citep{mucci1999papi}.  Therefore, an installation of \PAPI is required in 
order to use the package.  For convenience, we bundle \PAPI version \PAPIversion 
with \thispackage, which will install by default.  However, with appropriate 
configure arguments, one can easily build \thispackage with an existing system 
installation of \PAPI.  See Section~\ref{sec:installation} for details.

The current main features of \thispackage include:
\begin{enumerate}
  \item Simple, high-level interfaces that mimic \R's own profiling syntax.
  \item A low-level interface that mimics \PAPI's native calls, with  extremely 
  general functionality.
\end{enumerate}

Note that the \pkg{pbdPAPI} package is not officially affiliated with the 
\pkg{PAPI} project in any way.